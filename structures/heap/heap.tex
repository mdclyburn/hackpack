\section{Heap}\index{heap}\index{priority\_queue}
Heaps are very useful data structures that support at least the following operations:


\begin{itemize}
	\item Insert
	\item Either:
		\begin{itemize}
			\item Remove the smallest element 
			\item Remove the largest element
		\end{itemize}
\end{itemize}

Heaps that support removing the smallest element are called ``Min Heaps''
Heaps that support removing the largest element are called ``Max Heaps''
Many other implementations also implement a decrease key operation and delete operation.

The standard template library provides a two sets of functions that provide heaps.
By default, both implementations are ``Max Heaps'' but can be made to be ``Min Heaps''.
The first is the \lstinline{priority_queue} data structure found in the queue header.
The second is the \lstinline{make_heap} functions in found the algorithm header.
Neither of these implementations strictly implements the decrease key operation.
The code sample shown below shows how to create a Binary Min Heap as well in $O(N)$ time as the Heap Sort algorithm which runs in $O(N \log N)$ time.

\subsection{Reference}
\acmlisting[caption=Heap Reference, label=Heap Reference]{./structures/heap/heap.cpp}

\subsection{Applications}
\begin{itemize}
	\item Dijkstra's Algorithm
	\item Prim's Algorithm
	\item Priority Based Queuing
\end{itemize}
